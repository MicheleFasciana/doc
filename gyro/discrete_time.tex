%===================================================================
\chapter{Temporal Discretization}\label{chap.time}

The gyrokinetic treatment of electrons is particularly 
problematic in simulations because of the emergence of 
a host of numerical instabilities connected with the 
discretization of the electron parallel motion.  The 
presence of the potential in the advection term, 

$$
\frac{\partial h}{\partial t} + \frac{v_\parallel(\theta)}{q R_0}
\frac{\partial}{\partial\theta}\left( h - \alpha \dphi \right) 
+ \cdots = 0 \quad .
$$

\noindent
give rise to stability considerations much more troublesome 
than the simple electron parallel Courant limit.
Here, $\alpha = n_e/T_e$.  Physically, the parallel term
gives rise to the (pathological) high-frequency 
{\it electrostatic Alfv\'en} wave \cite{lee:2001}.  
At $\beta=0$, we have

$$
\omega_{\rm H} = \frac{k_\parallel}{k_r} 
\sqrt{\frac{m_i}{m_e}} \, \Omega_{ci} \quad .$$

\noindent
The frequency of this mode increases indefinitely as 
$m_e$ decreases, as also as $k_r$ decreases -- that is, 
as the radial box size grows.  To overcome this severe
and intrinsic simulation difficulty, we were finally 
presuaded to treat the electron advection implicitly

We experimented with a variety of recently-developed 
implicit-explicit (IMEX) Runge-Kutta (RK) schemes 
\cite{ascher:1995,ascher:1997,pareschi:2000,pareschi:2002,
kennedy:2003}, settling on the newest IMEX-SSP schemes of 
Pareschi and Russo \cite{pareschi:2002}.

%-----------------------------------------------------------------------
\subsection{Reduction to canonical form}

With all spatial operators discretized, the GKM system 
has the form
%
\begin{align}
{\dot h}_\si = &~\rhss(\{h_\si\},h_e) \, , 
  \quad \si = 1, \ldots, n_{\rm ion} 
  \label{imexi} \\
{\dot h}_e    = &~\rhse(\{h_\si\},h_e) + \rhsei(\{h_\si\},h_e) 
  \label{imexe}
\end{align}

where $\{h_\si\} = h_1,\ldots,h_{n_{\rm ion}}$.  In 
Eqs.~(\ref{imexi}) and (\ref{imexe}), the use of a tilde
on the RHS represents a {\it nonstiff} term, while the 
absence of a tilde indicates a {\it stiff} term.
The IMEX-RK schemes are written for equations in the 
canonical form
%
\begin{equation}
{\dot y} = {\widetilde Y}(y) + Y(y)
\end{equation}

with ${\widetilde Y}$ the explicit RHS and $Y$ the implicit RHS.
Thus, we have the connection
%
\begin{equation}
y = \begin{bmatrix} h_i \\ h_e \end{bmatrix} \quad 
{\widetilde Y} = \begin{bmatrix} \rhss \\ \rhse \end{bmatrix} \quad 
Y = \begin{bmatrix} 0 \\ \rhsei \end{bmatrix} 
\end{equation}

Now, there is some hidden complexity in how we have written 
the GKM equations, as the functions $H$ depend on the distribution 
function $h$ both directly, and indirectly through the fields.
The latter are given by 
%
\begin{align}
 \pop \cdot \dphi = &~\sum_\si z_\si \vop\left[\gop h_\si \right] - 
     \vop \left[ h_e - \alpha_e (\dphi-\vpe\dap) \right] \\
 \aop \cdot \dap  = &~\sum_\si z_\si \vop \left[\vp \, \gop h_\si \right] - 
         \vop \left[\vpe [h_e - \alpha_e (\dphi-\vpe\dap)] \right]
\end{align}

We have defined various operators in the Maxwell equations:
%
\begin{align}
\pop = &~\alpha_e + \sum_{\si=1}^{n_{\rm ion}} 
\alpha_\si z_\si^2 (1-\rop_\si) \\
\aop = &~-\frac{2\rhou^2}{\betau} \nabla_\perp^2 + 
\sum_{\si=1}^{n_{\rm ion}} \alpha_\si z_\si^2 \, \vop[\vp^2 \cdot]
\end{align}

It is only the fast electron advection term, $\rhsei$, which 
we wish to (or are able to) treat implicitly:

%
\begin{align}
\rhsei & = \frac{\vpe(r,\theta,\ehat)}{qR} 
           \frac{\partial}{\partial\theta} \left[
             h_e - \alpha_e \left( \dphi - \vp \dap \right) \right] \\
       & = \Omega(r,\lambda,\ehat) \, \frac{\partial}{\partial\tau} 
          \left[ h_e - \alpha_e \left( \dphi - \vp \dap \right) \right]
\end{align}

\subsection{IMEX-RK-SSP schemes of Pareschi and Russo}

The second-order IMEX schemes we consider can be summarized 
in Butcher tableau form:

%%
\begin{center}
\parbox{.25\linewidth}{
\mbox{Explicit} \par
\fbox{\begin{tabular}{|ccc}
0 & & \\
${\tilde a}_{21}$ & 0 & \\ 
${\tilde a}_{31}$ & ${\tilde a}_{32}$ & 0 \\ \hline
${\tilde w}_1$ & ${\tilde w}_2$ & ${\tilde w}_3$ 
\end{tabular}}}
\hskip 0.2in
\parbox{.25\linewidth}{
\mbox{Implicit} \par
\fbox{\begin{tabular}{|ccc}
$a_{11}$ & & \\
$a_{21}$ & $a_{22}$ & \\ 
$a_{31}$ & $a_{32}$ & $a_{33}$ \\ \hline
$w_1$ & $w_2$ & $w_3$ 
\end{tabular}}}
\end{center}

This notation corresponds to the following explicit 
evolution equations:

%
\begin{align}
y_1 & = y + \dt \, a_{11} Y_1 \\
y_2 & = y + \dt ( {\tilde a}_{21} \yt_1 + a_{21} Y_1 ) 
          + \dt \, a_{22} Y_2 \\ 
y_3 & = y + \dt ( {\tilde a}_{31} \yt_1 + {\tilde a}_{32} \yt_2 
                  + a_{31} Y_1 + a_{32} Y_2 ) 
          + \dt \, a_{33} Y_3 
\end{align}

%%
\begin{equation}
{\bar y} = y + \dt \sum_{k=1}^3 {\tilde w}_k \yt_k 
             + \dt \sum_{k=1}^3 w_k Y_k 
\label{newy}
\end{equation}

Above, $y$ is the old field vector (time $t$) and 
${\bar y}$ is the new field vector (time $t+\Delta t$).

In GYRO, we currently use the SSP2(3,2,2) scheme:

\begin{center}
SSP2(3,2,2) 
\parbox{.3\linewidth}{
\mbox{Explicit} \par
\fbox{\begin{tabular}{|ccc}
0 & & \\
0 & 0 & \\ 
0 & 1 & 0 \\ \hline
0 & $1/2$ & $1/2$
\end{tabular}}}
\parbox{.3\linewidth}{
\mbox{Implicit} \par
\fbox{\begin{tabular}{|ccc}
$1/2$  & & \\
$-1/2$ & $1/2$ & \\ 
 0   & $1/2$ & $1/2$  \\ \hline
 0   & $1/2$ & $1/2$ 
\end{tabular}}}
\end{center}

Since the 
diagonal coefficients of the implicit stages are 
equal, we need to solve the same implicit system at 
each stage.  We plan to implement the SSP2(3,3,2)
scheme, which requires the solution of two 
different implicit problems, but has half the 
advective Courant limit in explicit part.

\begin{center}SSP2(3,3,2)
\parbox{.3\linewidth}{
\mbox{Explicit} \par
\fbox{\begin{tabular}{|ccc}
0 & & \\
$1/2$ & 0 & \\ 
$1/2$ & $1/2$ & 0 \\ \hline
$1/3$ & $1/3$ & $1/3$
\end{tabular}}}
\parbox{.3\linewidth}{
\mbox{Implicit} \par
\fbox{\begin{tabular}{|ccc}
$1/4$ & & \\
0   & $1/4$ & \\ 
$1/3$ & $1/3$ & $1/3$ \\ \hline
$1/3$ & $1/3$ & $1/3$ 
\end{tabular}}}
\end{center}

\subsection{Implementation in GYRO}

At each integrator stage $(k=1,2,3)$ we are faced with:

%
\begin{align}
(h_\si)_k & = h_\si + \dt \sum_{p<k} {\tilde a}_{kp} (\rhss)_p 
\label{imex.ion}\\
(h_e)_k & = h_e + \dt \sum_{p<k} {\tilde a}_{kp} (\rhse)_p 
                + \dt \sum_{p<k} a_{kp} (\rhsei)_p + \dt \, 
                a_{kk} (\rhsei)_k
\label{imex.elec}
\end{align}


Because $\rhsei$ depends on the fields, we must advance the 
Maxwell equations at each stage before determining the 
$(h_e)_k$.  However, we can compute all $(h_\si)_k$ before 
solving the implicit problem.  First, solve formally for 
$(h_e)_k$:

%
\begin{equation}
(h_e)_k = (\delta h_e)_k + \alpha_e \frac{\cdto}{1+\cdto} 
\left( {\dphi} - \vpe {\dap} \right)_k 
\label{imex.he}
\end{equation}

where $(\delta h_e)_k$ is the explicitly-known quantity

%
\begin{equation}
(\delta h_e)_k = 
\frac{1}{1+\cdto} \left[ h_e + \dt \sum_{p<k} {\tilde a}_{pk} (\rhse)_k 
               + \dt \sum_{p<k} a_{pk} (\rhsei)_k \right ]
\label{imex.dhe}
\end{equation}

Substitute $(h_e)_k$ into the Maxwell equations:

%
\begin{align}
\pop \cdot (\dphi)_k & = \sum_{\si=1}^{n_{\rm ion}} 
   z_\si \vop\left[\gop (h_\si)_k \right] - 
        \vop \left[ (\delta h_e)_k \right] \notag \\
        & \qquad + \alpha_e \vop 
         \left[ \frac{1}{1+\cdto} (\dphi-\vpe\dap)_k \right] \\
~\notag \\
\aop \cdot ({\dap})_k & = \sum_{\si=1}^{n_{\rm ion}} 
 z_\si \vop \left[\vp \gop (h_s)_k \right] - 
         \vop \left[\vpe (\delta h_e)_k \right] \notag \\
        & \qquad + \alpha_e \vop
          \left[\vpe \frac{1}{1+\cdto} (\dphi-\vpe\dap)_k \right]
\end{align}

Multiply Maxwell equations by $F_m^{*i}(\theta)$ and operate 
with bounce-velocity summation operator $\fvop$ (noting 
that $\vop\vop=\vop$) to obtain 
$\dt$-coupled Poisson and Amp\`ere equations.
%
\begin{align}
(\pmat)_{m\mpr}^{i\ipr} {\ptil}_{\mpr}^{\ipr} & = 
  (\psrc)_m^i + {(\ipp)}^i_{m\mpr} {\ptil}_{\mpr}^i 
              + {(\ipa)}^i_{m\mpr} {\atil}_{\mpr}^i \\
(\amat)_{m\mpr}^{i\ipr} {\atil}_{\mpr}^{\ipr} & = 
  (\asrc)_m^i + {(\iap)}^i_{m\mpr} {\ptil}_{\mpr}^i  
              + {(\iaa)}^i_{m\mpr} {\atil}_{\mpr}^i
\end{align}

where we have omitted the stage number, $k$, for brevity. 
The field matrices are
%
\begin{align}
(\pmat)_{m\mpr}^{i\ipr} & = \fvop \left[ F^{*i}_m(\tq) 
              \pop^{i\ipr}(\tq) F^{\ipr}_{\mpr}(\tq) \right] \\
(\amat)_{m\mpr}^{i\ipr} & = \fvop \left[ F^{*i}_m(\tq)  
              \aop^{i\ipr}(\tq) F^{\ipr}_{\mpr}(\tq) \right] 
\end{align}

The coupling matrices are
%
\begin{align}
(\ipp)_{m\mpr}^i & = \alpha_e^i \fvop \left[
  F_m^{*i}(\tq) {\cal O}_{q q^\prime}^i F_{\mpr}^i(\tqp) \right] \\
(\ipa)_{m\mpr}^i & = - \alpha_e^i \fvop \left[
  F_m^{*i}(\tq) {\cal O}_{q q^\prime}^i F_{\mpr}^i(\tqp) \vpe(\tqp) \right] \\
(\iap)_{m\mpr}^i & =  \alpha_e^i \fvop \left[
  \vpe(\tq) F_m^{*i}(\tq) {\cal O}_{q q^\prime}^i F_{\mpr}^i(\tqp) \right] \\
(\iaa)_{m\mpr}^i & = - \alpha_e^i \fvop \left[
  \vpe(\tq) F_m^{*i}(\tq) {\cal O}_{q q^\prime}^i F_{\mpr}^i(\tqp)\vpe(\tqp) \right]
\end{align}

where the matrix ${\cal O}_{q q^\prime}$ is defined as

\begin{equation}
{\cal O}_{q q^\prime} (r,\lambda,\ehat) \doteq
{\left( \frac{1}{1+\cdto} \right)}_{q q^\prime} .
\end{equation}

The source matrices are 
%
\begin{align}
(\psrc)_m^i & = \fvop\left[ F_m^{*i}(\tq) H_P^i(\tq) \right] - 
        \fvop \left[ F_m^{*i}(\tq) \delta h_e^i(\tq) \right] 
\label{imex.sp}\\
(\asrc)_m^i & = \fvop\left[ F_m^{*i}(\tq) H_A^i(\tq) \right] - 
        \fvop \left[ \vpe(\tq) F_m^{*i}(\tq) \delta h_e^i(\tq) \right]
\label{imex.sa}
\end{align}

where 
%
\begin{equation}
H_P \doteq \sum_{\si=1}^{n_{\rm ion}} z_\si \gop h_\si 
\quad\mbox{and}\quad
H_A \doteq \sum_{\si=1}^{n_{\rm ion}} z_\si \vp \gop h_\si
\end{equation}

We reorganize the equations into a single partitioned system
suitable for solution with a direct sparse solver (the 
current implementation uses UMFPACK).
%
\begin{equation}
\begin{bmatrix} \pmat & 0 \\ 0 & \amat \end{bmatrix}
\begin{bmatrix} \ptil \\ \atil \end{bmatrix}
=
\begin{bmatrix} \psrc \\ \asrc \end{bmatrix}
+
\begin{bmatrix} \ipp(\dt) & \ipa(\dt) \\ 
                \iap(\dt) & \iaa(\dt)  \end{bmatrix}
\begin{bmatrix} \ptil \\ \atil \end{bmatrix}
\label{imex.field}
\end{equation}

Solve the system with UMFPACK ($\psrc$ and $\asrc$ need 
to be computed at each step). 

\subsection{Procedural summary}

Having precomputed the matrices $M$ and $I$, for 
$k=1,\ldots,3$ do
%
\begin{enumerate}
\item
Evaluate $(h_\si)_k$ using Eq.~(\ref{imex.ion}).
\item
Evaluate $(\delta h_e)_k$ using Eq.~(\ref{imex.dhe}).
\item
Evaluate $\psrc$ and $\asrc$ using Eq.~(\ref{imex.sp}) 
and Eq.~(\ref{imex.sa}), respectively.
\item
Solve Eq.~(\ref{imex.field}) for $\ptil$ and $\atil$.
\item
Evaluate $(h_e)_k$ using Eq.~(\ref{imex.he}). 
\end{enumerate}

With all stage values, $k=1,\ldots,3$, computed, 
evaluate ${\bar y} = ({\bar \{h_\si\}},{\bar h_e})^{\rm T}$ 
using Eq.~(\ref{newy}).  Then, synchronize the fields 
using an explicit field update.

\subsection{Time-Integration Considerations}

Ad-hoc semi-implicit schemes (for example, first 
order splitting) are normally a mixed-blessing,
offering improved stability of stiff terms at the expense 
of accuracy and/or stability loss for the explicit terms.
Higher order splittings can be constructed but 
are susceptible to a severe accuracy loss in the stiff limit.

The IMEX schemes we describe herein \cite{pareschi:2000} are:
\begin{itemize}
\item {\bf Asymptotic Preserving}:  the difference scheme 
is asymptotically correct in the stiff limit. 
\item {\bf Asymptotically Accurate}: the explicit 
integrator retains its order for the limiting 
differential-algebraic system in the stiff limit.
\item {\bf Strong-Stability-Preserving}: we recover an 
SSP-ERK \cite{gottlieb:2001} scheme in the stiff limit.
\end{itemize}

%-----------------------------------------------------------------------
