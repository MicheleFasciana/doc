%===================================================================
\chapter{Temporal Discretization}\label{chap.time}

The gyrokinetic treatment of electrons is problematic in simulations 
because of the emergence of a host of numerical instabilities connected 
with the discretization of the electron parallel motion.  The presence 
of the electrostatic and magnetic potentials in the advection term, 
%
\begin{equation}
\frac{\partial\hhat}{\partial\hat{t}} 
+ \frac{\vphat}{\mt q (R_0/a)} \frac{\partial}{\partial\theta}
\left( \hhat + z_a \alpha_a \hat{\Psi}_a \right) + \cdots
\end{equation}
%
for $a = e$ give rise to stability considerations much more troublesome 
than the simple electron parallel Courant limit.  Physically, the parallel 
term gives rise to the high-frequency {\it electrostatic Alfv\'en} 
wave \cite{lee:2001} at $\beta=0$.  In the slab limit, one can show

\begin{equation}
\omega_{\rm H} = \frac{k_\parallel}{k_r} 
\sqrt{\frac{m_i}{m_e}} \, \Omega_{ci} \; .
\end{equation}
%
This mode is physically pathological, because the frequency increases 
indefinitely as $m_e$ decreases, and also as $k_r$ decreases -- that is, 
as the radial box size grows.  While nonzero $\beta$ provides a cutoff,
the stable numerical treatment of this term is nontrivial.\footnote{The 
ability of Eulerian codes to treat this term both accurately and stably 
has been a major contributor to the relative success of Eulerian solvers 
in comparison to their PIC counterparts.}  To overcome this severe 
and intrinsic simulation difficulty, we were persuaded to treat the 
electron advection implicitly.  We experimented with a variety of 
implicit-explicit (IMEX) Runge-Kutta (RK) schemes that were popular in 
the period 2001-2003 
\cite{ascher:1995,ascher:1997,pareschi:2000,pareschi:2002,kennedy:2003}, 
settling on an IMEX-SSP scheme of Pareschi and Russo \cite{pareschi:2002}.

%-----------------------------------------------------------------------
\subsection{Reduction to canonical form}

For brevity, we suppress the toroidal harmonic index and abbreviate $h_{a,n}$ as
$h_a$.  With all spatial operators discretized, the GKM system has the form
%
\begin{align}
{\dot{h}}_a = &~\rhss(\{h_a\},h_e) \; , \quad a = 1, \ldots, n_{\rm ion} 
  \label{imexi} \\
{\dot{h}}_e = &~\rhse(\{h_a\},h_e) + \rhsei(\{\hhat\},h_e) \; , 
  \label{imexe}
\end{align}
%
where $\{h_a\} = h_1,\ldots, h_{n_{\rm ion}}$.  In Eqs.~(\ref{imexi}) 
and (\ref{imexe}), the use of a tilde on the RHS represents a {\it nonstiff} 
term, while the absence of a tilde indicates a {\it stiff} term.  The IMEX-RK 
schemes are written for equations in the canonical form
%
\begin{equation}
{\dot y} = {\widetilde Y}(y) + Y(y) \; ,
\end{equation}
%
with ${\widetilde Y}$ the explicit RHS and $Y$ the implicit RHS.
Thus, we have the connection
%
\begin{equation}
y = \begin{bmatrix} \{ \hat{h}_a \} \\ h_e \end{bmatrix} \quad 
{\widetilde Y} = \begin{bmatrix} \{\rhss\} \\ \rhse \end{bmatrix} \quad 
Y = \begin{bmatrix} 0 \\ \rhsei \end{bmatrix} \; . 
\end{equation}
%
Now, there is some hidden complexity in how we have written 
the GKM equations, as the functions $H$ depend on the distribution 
function $h$ both directly, and indirectly through the fields.
We emphasize that it is only the fast electron advection term, $\rhsei$, 
which we wish to (or are able to) treat implicitly:
%
\begin{align}
H_{e} \doteq &~-\hat{v}_{\parallel e}(r,\theta,\varepsilon)
\frac{a}{q R_0 G_{\theta}} \frac{\partial}{\partial \theta}
\left[ \hat{h}_e - \alpha_e \left( \delta \hat{\phi} - \hat{v}_{\parallel e} 
\delta \hat{A}_{\parallel} - \varepsilon \lambda \hat{T}_{e} \delta 
\hat{B}_{\parallel} \right) \right] \\
= &~-\Omega(r,\theta,\varepsilon) \partial_\tau
\left[ \hat{h}_{e} - \alpha_e \left( \delta \hat{\phi} 
- \hat{v}_{\parallel e} \delta \hat{A}_{\parallel} 
- \varepsilon \lambda \hat{T}_{e} \delta \hat{B}_{\parallel} \right) \right] \; ,
\end{align}
%
where from this point forward we suggest that the reader consider the 
object $\partial_\tau$ as a matrix.

\subsection{IMEX-RK-SSP schemes of Pareschi and Russo}

The second-order IMEX schemes we consider can be summarized in Butcher 
tableau form as
%
\begin{center}
\parbox{.25\linewidth}{
\mbox{Explicit} \par
\fbox{\begin{tabular}{|ccc}
0 & & \\
${\tilde a}_{21}$ & 0 & \\ 
${\tilde a}_{31}$ & ${\tilde a}_{32}$ & 0 \\ \hline
${\tilde w}_1$ & ${\tilde w}_2$ & ${\tilde w}_3$ 
\end{tabular}}}
\hskip 0.2in
\parbox{.25\linewidth}{
\mbox{Implicit} \par
\fbox{\begin{tabular}{|ccc}
$a_{11}$ & & \\
$a_{21}$ & $a_{22}$ & \\ 
$a_{31}$ & $a_{32}$ & $a_{33}$ \\ \hline
$w_1$ & $w_2$ & $w_3$ 
\end{tabular}}}
\end{center}
%
This notation corresponds to the following explicit evolution equations:
%
\begin{align}
y_1 & = y + \dt \, a_{11} Y_1 \\
y_2 & = y + \dt ( {\tilde a}_{21} \yt_1 + a_{21} Y_1 ) 
          + \dt \, a_{22} Y_2 \\ 
y_3 & = y + \dt ( {\tilde a}_{31} \yt_1 + {\tilde a}_{32} \yt_2 
                  + a_{31} Y_1 + a_{32} Y_2 ) 
          + \dt \, a_{33} Y_3 
\end{align}
%
\begin{equation}
{\bar y} = y + \dt \sum_{k=1}^3 {\tilde w}_k \yt_k 
             + \dt \sum_{k=1}^3 w_k Y_k \; .
\label{newy}
\end{equation}
%
Above, $y$ is the old field vector (time $t$) and ${\bar y}$ is the new 
field vector (time $t+\Delta t$).  In GYRO, we currently use the SSP2(3,2,2) 
scheme:
%
\begin{center}
SSP2(3,2,2) 
\parbox{.3\linewidth}{
\mbox{Explicit} \par
\fbox{\begin{tabular}{|ccc}
0 & & \\
0 & 0 & \\ 
0 & 1 & 0 \\ \hline
0 & $1/2$ & $1/2$
\end{tabular}}}
\parbox{.3\linewidth}{
\mbox{Implicit} \par
\fbox{\begin{tabular}{|ccc}
$1/2$  & & \\
$-1/2$ & $1/2$ & \\ 
 0   & $1/2$ & $1/2$  \\ \hline
 0   & $1/2$ & $1/2$ 
\end{tabular}}}
\end{center}
%
Since the diagonal coefficients of the implicit stages are equal, we need 
to solve the same implicit system at each stage.  It would be good at some 
point to implement the SSP2(3,3,2) scheme, which requires the solution of 
two different implicit problems, but has half the advective Courant limit 
in explicit part.
%
\begin{center}SSP2(3,3,2)
\parbox{.3\linewidth}{
\mbox{Explicit} \par
\fbox{\begin{tabular}{|ccc}
0 & & \\
$1/2$ & 0 & \\ 
$1/2$ & $1/2$ & 0 \\ \hline
$1/3$ & $1/3$ & $1/3$
\end{tabular}}}
\parbox{.3\linewidth}{
\mbox{Implicit} \par
\fbox{\begin{tabular}{|ccc}
$1/4$ & & \\
0   & $1/4$ & \\ 
$1/3$ & $1/3$ & $1/3$ \\ \hline
$1/3$ & $1/3$ & $1/3$ 
\end{tabular}}}
\end{center}

\subsection{Implementation in GYRO}

The complication which arisese when applying these methods to GYRO are 
connected with the field solve.  For all the so-called SDIRK methods, 
the implicit equations at each stage have a generic form.  In the 
present case, at each integrator stage $(k=1,2,3)$, we are faced with:
%
\begin{align}
(h_\si)_k & = h_\si + \dt \sum_{p<k} {\tilde a}_{kp} (\rhss)_p 
\label{imex.ion} \; , \\
(h_e)_k & = h_e + \dt \sum_{p<k} {\tilde a}_{kp} (\rhse)_p 
                + \dt \sum_{p<k} a_{kp} (\rhsei)_p + \dt \, 
                a_{kk} (\rhsei)_k \; .
\label{imex.elec}
\end{align}
%
Because $\rhsei$ depends on the fields, we must advance the Maxwell 
equations at each stage before determining the $(h_e)_k$.  However, 
we can compute all $(h_a)_k$ for $a=1,\ldots,n_{\rm ion}$ before solving 
the implicit problem.  First, solve formally for $(h_e)_k$:
%
\begin{equation}
(h_e)_k = (\delta h_e)_k + \alpha_e \frac{\cdto}{1+\cdto} 
\left( \delta \hat{\phi} 
- \hat{v}_{\parallel e} \delta \hat{A}_{\parallel} 
- \varepsilon \lambda \hat{T}_{e} \delta \hat{B}_{\parallel} \right)_k 
\label{imex.he}
\end{equation}
%
where $(\delta h_e)_k$ is the explicitly-known quantity
%
\begin{equation}
(\delta h_e)_k = 
\frac{1}{1+\cdto} \left[ h_e + \dt \sum_{p<k} {\tilde a}_{kp} (\rhse)_k 
          + \dt \sum_{p<k} a_{kp} (\rhsei)_k \right ] \; .
\label{imex.dhe}
\end{equation}
% 
Now, substitute the analytic expression for $(h_e)_k$ into the Maxwell 
equations.  This substitution gives a version of the Maxwell equations 
that looks like the standard explicit forms, but with added field 
terms arising from the second term on the RHS of Eq.~(\ref{imex.he}).
If we multiply the Maxwell equations by $F_m^{*i}(\theta)$, and operate 
with the bounce-velocity summation operator $\fvop$ (noting that 
$\vop\vop=\vop$), we obtain $\dt$-coupled Poisson-Amp\`ere equations.
%
\begin{align}
(M_{PP})_{m\mpr}^{i\ipr} {\ptil}_{\mpr}^{\ipr} 
   + (M_{PB})_{m\mpr}^{i\ipr}  {\btil}_{\mpr}^{\ipr} & = 
  (S_P)_m^i + {(\ipp)}^i_{m\mpr} {\ptil}_{\mpr}^i 
              + {(\ipa)}^i_{m\mpr} {\atil}_{\mpr}^i 
              + {(\ipb)}^i_{m\mpr} {\btil}_{\mpr}^i \\
(M_{AA})_{m\mpr}^{i\ipr} {\atil}_{\mpr}^{\ipr} & = 
  (S_A)_m^i + {(\iap)}^i_{m\mpr} {\ptil}_{\mpr}^i  
              + {(\iaa)}^i_{m\mpr} {\atil}_{\mpr}^i
              + {(\iab)}^i_{m\mpr} {\btil}_{\mpr}^i \\
(M_{BP})_{m\mpr}^{i\ipr} {\ptil}_{\mpr}^{\ipr} 
   + (M_{BB})_{m\mpr}^{i\ipr}  {\btil}_{\mpr}^{\ipr} & = 
  (S_B)_m^i + {(\ibp)}^i_{m\mpr} {\ptil}_{\mpr}^i 
              + {(\iba)}^i_{m\mpr} {\atil}_{\mpr}^i 
              + {(\ibb)}^i_{m\mpr} {\btil}_{\mpr}^i 
\end{align}
%
Here the \textit{field matrices} are:
\begin{align}
(M_{PP})_{m\mpr}^{ii\prime} = &~\fvop\left[ F_{m}^{*i}(\theta(\tau_q))
{\cal L}_{PP}^{ii\prime}(\theta(\tau_q))F_{m\prime}^{i\prime}(\theta(\tau_q)) \right] \\
(M_{PB})_{m\mpr}^{ii\prime} = &~\fvop\left[ F_{m}^{*i}(\theta(\tau_q))
{\cal L}_{PB}^{ii\prime}(\theta(\tau_q)) F_{m\prime}^{i\prime}(\theta(\tau_q)) \right] \\
(M_{AA})_{m\mpr}^{ii\prime} = &~\fvop\left[ F_{m}^{*i}(\theta(\tau_q))
{\cal L}_{AA}^{ii\prime}(\theta(\tau_q)) F_{m\prime}^{i\prime}(\theta(\tau_q)) \right] \\
(M_{BB})_{m\mpr}^{ii\prime} = &~\fvop\left[ F_{m}^{*i}(\theta(\tau_q))
{\cal L}_{BB}^{ii\prime}(\theta(\tau_q)) F_{m\prime}^{i\prime}(\theta(\tau_q)) \right] \\
(M_{BP})_{m\mpr}^{ii\prime} = &~\fvop\left[ F_{m}^{*i}(\theta(\tau_q))
{\cal L}_{BP}^{ii\prime}(\theta(\tau_q)) F_{m\prime}^{i\prime}(\theta(\tau_q)) \right]
\end{align}
%
where
%
\begin{align}
{\cal L}_{PP} = &~\bar{\lambda}_{D}^2 \nabla_\perp^2 + \alpha_{e} + \sum_{a=1}^{n_{ion}} 
\alpha_{a} z_{a}^{2} V\left[ 1-{\cal G}_{0a}^{2} \right] \\
{\cal L}_{PB} = &~\sum_{a=1}^{n_{ion}} -2 z_{a} \hat{n}_{a}
V[\varepsilon \lambda {\cal G}_{0a} {\cal G}_{1a} ] \\
{\cal L}_{AA} = &~\frac{- 2 {\bar \rho}_{s,{\rm unit}}^{2}}{\bar{\beta}_{e,{\rm unit}}}
\nabla_{\perp}^{2} + \sum_{a=1}^{n_{ion}} \alpha_{a} z_{a}^{2}
V[\hat{v}_{\parallel}^{2}{\cal G}_{0a}^{2}] \\
{\cal L}_{BB} = &~\frac{1}{\bar{\beta}_{e,{\rm unit}}} +  
\sum_{a=1}^{n_{ion}} 2  \hat{n}_{a} \hat{T}_{a}
V[\lambda^{2} {\cal G}_{1a}^{2} \varepsilon^{2} ] \\
{\cal L}_{BP} = &~\sum_{a=1}^{n_{ion}} z_{a} \hat{n}_{a}
V[\varepsilon \lambda {\cal G}_{1a} {\cal G}_{0a} ] \; .
\end{align}
%
The \textit{advection matrices} are:
%
\begin{eqnarray}
%
(I_{PP})_{mm\prime}^{i} & = & \alpha_{e}^{i} {\cal F} V\left[ 
F_{m}^{*i}(\theta(\tau_q)) {\cal O}_{qq\prime}^{i} 
F_{m\prime}^{i}(\theta(\tau_{q\prime})) \right] \\
%
(I_{PA})_{mm\prime}^{i} & = & -\alpha_{e}^{i} {\cal F} V\left[ 
F_{m}^{*i}(\theta(\tau_q)) {\cal O}_{qq\prime}^{i} 
F_{m\prime}^{i}(\theta(\tau_{q\prime})) 
\hat{v}_{\parallel e}(\theta(\tau_{q\prime})) \right] \\
%
(I_{PB})_{mm\prime}^{i} & = & -\alpha_{e}^{i} \hat{T}_{e}^{i} {\cal F} V\left[ 
F_{m}^{*i}(\theta(\tau_q)) {\cal O}_{qq\prime}^{i} 
F_{m\prime}^{i}(\theta(\tau_{q\prime})) \varepsilon \lambda  \right] \\
%
(I_{AP})_{mm\prime}^{i} & = & \alpha_{e}^{i} {\cal F} V\left[ 
\hat{v}_{\parallel e}(\theta(\tau_{q}))
F_{m}^{*i}(\theta(\tau_q)) {\cal O}_{qq\prime}^{i} 
F_{m\prime}^{i}(\theta(\tau_{q\prime})) \right]\\
%
(I_{AA})_{mm\prime}^{i} & = & -\alpha_{e}^{i} {\cal F} V\left[ 
\hat{v}_{\parallel e}(\theta(\tau_{q}))
F_{m}^{*i}(\theta(\tau_q)) {\cal O}_{qq\prime}^{i} 
F_{m\prime}^{i}(\theta(\tau_{q\prime})) 
\hat{v}_{\parallel e}(\theta(\tau_{q\prime}))
\right]\\
%
(I_{AB})_{mm\prime}^{i} & = & -\alpha_{e}^{i} \hat{T}_{e}^{i} {\cal F} V\left[ 
\hat{v}_{\parallel e}(\theta(\tau_q))
F_{m}^{*i}(\theta(\tau_q)) {\cal O}_{qq\prime}^{i} 
F_{m\prime}^{i}(\theta(\tau_{q\prime}))  \varepsilon \lambda \right]\\
%
(I_{BP})_{mm\prime}^{i} & = & \frac{1}{2} \alpha_{e}^{i} \hat{T}_{e}^{i}
{\cal F} V\left[ 
F_{m}^{*i}(\theta(\tau_q)) {\cal O}_{qq\prime}^{i} 
F_{m\prime}^{i}(\theta(\tau_{q\prime})) \varepsilon \lambda  \right]\\
%
(I_{BA})_{mm\prime}^{i} & = & -\frac{1}{2} \alpha_{e}^{i} \hat{T}_{e}^{i}
{\cal F} V\left[ 
F_{m}^{*i}(\theta(\tau_q)) {\cal O}_{qq\prime}^{i} 
F_{m\prime}^{i}(\theta(\tau_{q\prime})) 
\hat{v}_{\parallel e}(\theta(\tau_{q\prime}))
\varepsilon \lambda  \right]\\
%
(I_{BB})_{mm\prime}^{i} & = & -\frac{1}{2} \alpha_{e}^{i} (\hat{T}_{e}^{i})^2
{\cal F} V\left[ 
F_{m}^{*i}(\theta(\tau_q)) {\cal O}_{qq\prime}^{i} 
F_{m\prime}^{i}(\theta(\tau_{q\prime}))  \varepsilon^2 \lambda^2 \right]
\end{eqnarray}
%
Next, the \textit{source matrices} are
%
\begin{align}
(S_{P})_{m}^{i} = &~\fvop\left[ F_{m}^{*i}(\theta(\tau_q)) 
H_{P}^{i}(\theta(\tau_q)) \right] - \fvop\left[ F_{m}^{*i}(\theta(\tau_q)) 
\delta \hat{h}_{e}^{i}(\theta(\tau_q)) \right] 
 \label{source.p}\\
(S_{A})_{m}^{i} = &~\fvop\left[ F_{m}^{*i}(\theta(\tau_q)) 
H_{A}^{i}(\theta(\tau_q)) \right]-\fvop\left[ \hat{v}_{\parallel e}(\theta(\tau_q)) 
F_{m}^{*i}(\theta(\tau_q)) \delta \hat{h}_{e}^{i}(\theta(\tau_q)) \right] 
\label{source.a}\\
(S_{B})_{m}^{i} = &~\fvop\left[ F_{m}^{*i}(\theta(\tau_q)) H_{B}^{i}(\theta(\tau_q)) 
\right] - \frac{1}{2} \hat{T}_{e} 
\fvop\left[\varepsilon \lambda F_{m}^{*i}(\theta(\tau_q)) 
\delta \hat{h}_{e}^{i}(\theta(\tau_q)) \right]
\label{source.b}
\end{align}
%
where
%
\begin{align}
H_{P} \doteq &~\sum_{a=1}^{n_{ion}} z_{a} {\cal G_0}_{a} 
\hat{h}_{a} \\
H_{A} \doteq&~\sum_{a=1}^{n_{ion}} z_{a} \hat{v}_{\parallel} {\cal G_0}_{a} 
\hat{h}_{a} \\
H_{B} \doteq&~\sum_{a=1}^{n_{ion}} -\hat{T}_{a} 
{\cal G_1}_{a} \hat{h}_{a} \varepsilon \lambda \; .
\end{align}
%
In the preceeding expressions, we have omitted the stage number, $k$, for 
brevity.  The advection matrix ${\cal O}_{q q^\prime}$ is defined as
%
\begin{equation}
{\cal O}_{q q^\prime} (r,\lambda,\ehat) \doteq
{\left( \frac{1}{1+\cdto} \right)}_{q q^\prime} .
\end{equation}
%
We reorganize the equations into a single partitioned system
suitable for solution with a direct sparse solver (the 
current implementation uses UMFPACK).  Solve the system 
with UMFPACK ($S_P$, $S_A$ and $S_B$ need to be computed at each step). 
%
\begin{equation}
\left[ \begin{array}{ccc}
M_{PP} & 0      & M_{PB} \\
0      & M_{AA} & 0 \\
M_{BP} & 0      & M_{BB} 
\end{array} \right]
\left[ \begin{array}{c}
\ptil \\ \atil \\ \btil 
\end{array} \right]
= \left[ \begin{array}{c}
S_{P} \\ S_{A} \\ S_{B}
\end{array} \right]
+ \left[ \begin{array}{ccc}
I_{PP} & I_{PA} & I_{PB} \\
I_{AP} & I_{AA} & I_{AB} \\
I_{BP} & I_{BA} & I_{BB}
\end{array} \right]
\left[ \begin{array}{c}
\ptil \\ \atil \\ \btil 
\end{array} \right]
\label{imex.field}
\end{equation}

\subsection{Procedural summary}

Having precomputed the matrices $M$ and $I$, for $k=1,\ldots,3$:
%
\begin{enumerate}
\item
Evaluate $(h_\si)_k$ using Eq.~(\ref{imex.ion}).
\item
Evaluate $(\delta h_e)_k$ using Eq.~(\ref{imex.dhe}).
\item
Evaluate $S_P$, $S_A$ and $S_B$ using Eqs.~(\ref{source.p}) through
(\ref{source.b}), respectively.
\item
Solve Eq.~(\ref{imex.field}) for $\ptil$, $\atil$ and $\btil$.
\item
Evaluate $(h_e)_k$ using Eq.~(\ref{imex.he}). 
\end{enumerate}
%
With all stage values, $k=1,2,3$, computed, evaluate 
${\bar y} = ({\{\bar{h}_\si\}},{\bar h_e})^{\rm T}$ using 
Eq.~(\ref{newy}).  Then, synchronize the fields using an explicit 
field update.

\subsection{Time-Integration Considerations}

Ad-hoc semi-implicit schemes (for example, first 
order splitting) are normally a mixed-blessing,
offering improved stability of stiff terms at the expense 
of accuracy and/or stability loss for the explicit terms.
Higher order splittings can be constructed but 
are susceptible to a severe accuracy loss in the stiff limit.
%
The IMEX schemes we describe herein \cite{pareschi:2000} are:
%
\begin{itemize}
\item {\bf Asymptotic Preserving}:  the difference scheme 
is asymptotically correct in the stiff limit. 
\item {\bf Asymptotically Accurate}: the explicit 
integrator retains its order for the limiting 
differential-algebraic system in the stiff limit.
\item {\bf Strong-Stability-Preserving}: we recover an 
SSP-ERK \cite{gottlieb:2001} scheme in the stiff limit.
\end{itemize}

%-----------------------------------------------------------------------
